

\chapter{Bibliographic references} \label{OPERATION_SCE}
%This chapter presents in detail the operational modes corresponding to each proposed spatial layout. The operational modes describe how the robot behaves considering its internal connections and limitations. This chapter is organized into two main sections:
%\begin{itemize}
%    \item Section~\ref{NORMAL_SCE} - Normal operational modes: describe the operations/tasks that the robot can perform considering its structure (e.g.: recording images of the whole platform, inspecting equipment vibration during pre-programmed period, climb the rails with a set speed, etc.).
%    \item Section~\ref{FAILURE_SCE} - Failure operational modes: describe what changes in the robot normal operation upon a failure of any nature. Thus, to describe the failure operational modes, our methodology is to:
%        \begin{itemize}
%            \item Analyze the robot electronics architecture limitations.
%            \item Outline the events that can be initiated in case of failure.
%            \item Define possible actions/interventions that the robot or the operator can take.
%            \item Propose solutions to circumvent problems brought up due the limitations of the analyzed architecture.
%            \item Propose a new and more suitable electronics architecture, and, hence, a new spatial layout.
%        \end{itemize}
%\end{itemize}
%\newpage
%
%
%%%******************************************************************************
%%% SECTION - Section
%%%******************************************************************************
%
%
%\section{Normal Operational Modes} \label{NORMAL_SCE}
%To be written.
%
%
%\section{Failure Operational Modes} \label{FAILURE_SCE}
%Failure modes for all Spatial Layouts:\\ \textbf{\href{http://www.coep.ufrj.br:8080/xwiki/bin/download/DORIS/G2+Miscellaneous+Files/Cen\%C3\%A1rios\%20de\%20Falha\%20-\%20ingles.pdf}{DORIS XWiki Repository Link.}}
%\newline
%\newline
%Failure modes for individual failures of equipments, considering Spatial Layout 1:\\
%\textbf{\href{http://www.coep.ufrj.br:8080/xwiki/bin/download/DORIS/G2+Miscellaneous+Files/Tabela\%20Falhas.pdf}{DORIS XWiki Repository Link.}} 